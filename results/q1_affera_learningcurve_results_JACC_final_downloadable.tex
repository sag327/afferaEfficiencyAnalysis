
\documentclass[12pt]{article}
\usepackage[margin=1in]{geometry}
\usepackage{booktabs}
\usepackage{array}
\usepackage{lscape}

\begin{document}

\section*{Statistical Methods}

We performed a retrospective observational analysis of atrial fibrillation (AF) ablation procedures to evaluate the independent effect of the Affera pulsed-field ablation (PFA) system on total procedure duration and to quantify any Affera-specific learning curve. The analytic cohort used a prespecified baseline window of 180 days prior to Affera adoption (baselineDays = 180). Operators were included if they contributed at least 15 AF ablations in both the baseline and Affera eras (min\_cases\_per\_group = 15). 

The final dataset consisted of three case categories: (1) pre-Affera non-PFA ablations performed during the baseline window, (2) Affera PFA ablations after adoption, and (3) post-Affera non-PFA ablations performed after Affera introduction. This design allowed separation of Affera adoption effects from secular time trends.

Procedure duration (minutes) was modeled using a linear mixed-effects model with the natural logarithm of duration as the dependent variable. Fixed effects included: Affera use (PFA vs.\ non-PFA), procedural complexity (PVI-only vs.\ PVI-plus additional ablation [PVI+]), calendar time in days since the first included case, an Affera-by-complexity interaction to assess whether Affera’s effect differed between PVI and PVI+ cases, and a centered Affera case index to quantify Affera-specific learning with increasing operator experience. Random intercepts were included for operators to account for between-operator differences in baseline duration. Effects are reported as log-scale coefficients, exponentiated percent changes, 95\% confidence intervals (CIs), and two-sided p-values.

Because Affera case index and calendar time may correlate, we conducted prespecified collinearity diagnostics including Pearson correlations among fixed-effect predictors and variance inflation factors (VIFs). Correlations $|r|>0.7$ generated a diagnostic warning, while VIF values were interpreted using standard thresholds (moderate concern for VIF$>5$, serious concern for VIF$>10$).

\section*{Results}

A total of 1,752 AF ablation procedures performed by 11 operators met inclusion criteria, including 513 pre-Affera non-PFA cases, 1,014 Affera cases, and 225 post-Affera non-PFA cases. Of these, 617 procedures were PVI-only and 1,135 were PVI+. Unadjusted procedure durations by era and complexity are summarized in Table~\ref{tab:durations}.

\begin{table}[h!]
\centering
\begin{tabular}{lccc}
\toprule
\textbf{Group} & \textbf{n} & \textbf{Mean (min)} & \textbf{Median (min)} \\
\midrule
Pre-Affera non-PFA (all)   & 513  & 97.5 & 92.0 \\
Pre-Affera non-PFA (PVI)   & 180  & 84.5 & 80.0 \\
Pre-Affera non-PFA (PVI+)  & 333  & 104.5 & 98.0 \\
Post-Affera non-PFA (all)  & 225  & 87.5 & 84.0 \\
Post-Affera non-PFA (PVI)  & 76   & 71.9 & 74.0 \\
Post-Affera non-PFA (PVI+) & 149  & 95.3 & 88.0 \\
Affera (all)               & 1,014 & 74.2 & 69.0 \\
Affera (PVI)               & 361  & 65.6 & 62.0 \\
Affera (PVI+)              & 653  & 78.9 & 73.0 \\
\bottomrule
\end{tabular}
\caption{Unadjusted procedure durations for baseline, post-Affera non-PFA, and Affera cases.}
\label{tab:durations}
\end{table}

\clearpage
\begin{landscape}
\begin{table}[h!]
\centering
\begin{tabular}{lcccc}
\toprule
\textbf{Effect} & \textbf{Estimate (log scale)} & \textbf{95\% CI} & \textbf{Percent Change} & \textbf{p-value} \\
\midrule
Affera use (PVI) & -0.122 & [-0.188, -0.057] & -11.5\% & 0.000273 \\
Affera effect in PVI+ (additional) & -0.064 & [-0.128, -0.001] & -6.2\% & 0.0475 \\
Calendar time (per day) & -0.000 & [-0.001, -0.000] & -0.0\% per day & $4.78\times10^{-6}$ \\
PVI+ vs.\ PVI & 0.281 & [0.231, 0.331] & +32.4\% & $4.18\times10^{-27}$ \\
Affera learning (per Affera case) & -0.001 & [-0.002, -0.000] & -0.11\% per case & 0.00224 \\
\bottomrule
\end{tabular}
\caption{Mixed-effects model estimates for log-transformed procedure duration. Affera effects are referenced to non-PFA PVI-only procedures; the additional PVI+ effect represents the Affera-by-complexity interaction.}
\label{tab:mixedmodel}
\end{table}
\end{landscape}

In adjusted analyses (Table~\ref{tab:mixedmodel}), Affera use was associated with an 11.5\% reduction in PVI procedure duration compared with non-PFA PVI-only cases (95\% CI: 5.5\% to 17.2\%; p=0.000273). The Affera benefit was larger for complex PVI+ procedures, with an additional 6.2\% reduction beyond the PVI effect (95\% CI: 0.1\% to 12.0\%; p=0.0475), yielding an overall estimated reduction of approximately 17.7\% for PVI+ cases. Procedural complexity independently increased duration: PVI+ procedures were 32.4\% longer than PVI-only cases (p=$4.18\times10^{-27}$).

Across all AF ablations (combining PVI-only and PVI+ procedures in their observed proportions), the Affera system was associated with an estimated 15.0\% reduction in total procedure duration (95\% CI: 8.2\% to 21.3\% reduction). This overall effect was derived from the mixed-effects model by weighting the Affera effect in PVI-only cases and the larger Affera effect in PVI+ cases according to their distribution in the dataset (36\% PVI-only, 64\% PVI+). The resulting estimate provides a single summary measure of the procedural efficiency benefit attributable to Affera across the full spectrum of AF ablation complexity.

The Affera learning-curve term was statistically significant but modest, corresponding to a 0.11\% reduction in duration per additional Affera case (95\% CI: 0.04\% to 0.17\%; p=0.00224). This translates to an approximate 2.0\% reduction between an operator’s first and twentieth Affera cases. After accounting for Affera adoption and Affera-specific learning, the residual secular time trend was near zero in magnitude and not clinically meaningful, despite statistical detectability (p=$4.78\times10^{-6}$).

Collinearity diagnostics identified high correlation between Affera use and calendar time ($r=0.74$), reflecting adoption timing. However, VIF values were low (2.58 for Affera use; 2.91 for calendar time; 1.33 for Affera index), indicating no meaningful multicollinearity and stable estimation of fixed effects.

\section*{Conclusions}

In a mixed-effects framework that incorporated Affera-specific learning and leveraged post-Affera non-PFA cases to identify temporal trends, Affera PFA was associated with clinically meaningful reductions in AF ablation duration. PVI-only procedures were approximately 12\% faster with Affera, and complex PVI+ procedures were approximately 18\% faster. The Affera learning curve was statistically present but small (about a 2\% improvement over the first 20 Affera cases). After modeling adoption and learning, no clinically relevant secular improvement over calendar time remained, supporting the interpretation that Affera use was the primary driver of observed efficiency gains.

\end{document}

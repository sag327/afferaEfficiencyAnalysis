
\documentclass[12pt]{article}
\usepackage[margin=1in]{geometry}
\usepackage{booktabs}
\usepackage{siunitx}
\usepackage{array}
\usepackage{longtable}

\begin{document}

\section*{Statistical Methods}

We performed a retrospective observational analysis of atrial fibrillation (AF) ablation procedures to evaluate the effect of the Affera pulsed-field ablation (PFA) system on total procedure duration. The analytic cohort included all non–PFA AF ablations performed in the six months prior to Affera adoption (pre-Affera baseline), all Affera cases performed after adoption, and all non–PFA AF ablations performed after Affera introduction. Inclusion of both pre- and post-Affera non–PFA procedures allowed estimation of secular efficiency trends and separation of technology-specific effects from temporal changes in practice.

Procedure duration was modeled using a linear mixed-effects framework with the natural logarithm of duration as the dependent variable. Fixed effects included Affera use, procedural complexity (pulmonary vein isolation [PVI] vs.\ PVI with additional ablation [PVI+]), calendar time, and an interaction term assessing whether the Affera effect differed between PVI and PVI+ procedures. Operator identity was included as a random intercept to account for baseline differences in operator efficiency.

\section*{Results}

A total of 1,752 AF ablation procedures performed by 11 operators met inclusion criteria, including 513 pre-Affera non–PFA procedures, 225 post-Affera non–PFA procedures, and 1,014 Affera procedures. Of these, 617 were PVI-only and 1,135 were PVI+.

\subsection*{Procedure Durations}

\begin{table}[h!]
\centering
\begin{tabular}{lccc}
\toprule
\textbf{Group} & \textbf{n} & \textbf{Mean (min)} & \textbf{Median (min)} \\
\midrule
Pre-Affera non–PFA (all) & 513 & 97.5 & 92.0 \\
Pre-Affera non–PFA (PVI) & 180 & 84.5 & 80.0 \\
Pre-Affera non–PFA (PVI+) & 333 & 104.5 & 98.0 \\
Post-Affera non–PFA (all) & 225 & 87.5 & 84.0 \\
Post-Affera non–PFA (PVI) & 76 & 71.9 & 74.0 \\
Post-Affera non–PFA (PVI+) & 149 & 95.3 & 88.0 \\
Affera (all) & 1,014 & 74.2 & 69.0 \\
Affera (PVI) & 361 & 65.6 & 62.0 \\
Affera (PVI+) & 653 & 78.9 & 73.0 \\
\bottomrule
\end{tabular}
\end{table}

\subsection*{Mixed-Effects Model Estimates}

\begin{table}[h!]
\centering
\begin{tabular}{lccc}
\toprule
\textbf{Effect} & \textbf{Estimate (log scale)} & \textbf{95\% CI} & \textbf{Percent Change} \\
\midrule
Affera use (PVI) & -0.085 & [-0.146, -0.023] & -8.1\% \\
Affera effect in PVI+ (additional) & -0.064 & [-0.128, -0.000] & -6.2\% \\
Calendar time (per day) & -0.001 & [-0.001, -0.000] & -0.1\% per day \\
PVI+ vs.\ PVI & 0.280 & [0.230, 0.330] & +32.3\% \\
\bottomrule
\end{tabular}
\end{table}

After adjustment for operator-level variation and secular time trends, use of the Affera PFA system was associated with an 8.1\% reduction in procedure duration for PVI-only cases (95\% CI: 2.3\% to 13.6\%). PVI+ procedures were 32\% longer than PVI procedures overall (95\% CI: 26\% to 39\%). The combined effect of Affera for PVI+ procedures corresponded to a 14.3\% reduction in duration, representing an expected savings of approximately 12–15 minutes for a typical complex ablation. The secular trend indicated a 0.1\% daily improvement in procedure duration, corresponding to roughly 27–30\% annual gains independent of technology.

Weighting by the distribution of PVI and PVI+ procedures, the overall adjusted effect of Affera across all AF ablations was a 12.1\% reduction in procedure duration.

\section*{Conclusion}

After accounting for secular improvements in laboratory efficiency and operator-level differences, the Affera PFA system was associated with meaningful reductions in AF ablation duration, particularly for more complex PVI+ procedures.

\end{document}
